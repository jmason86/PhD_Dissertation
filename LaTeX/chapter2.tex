\chapter{Relevant Background}
\label{chapterbackground}

\section{Solar Corona}

\section{Physics of Solar Eruptive Event Initiation}

\section{Space Weather}

\section{EUV Emission}
Maxwellian plasma versus not. Focus herein will be on Maxwellian plasmas as the studies don't focus on super hot plasmas. 

\section{Instrument Descriptions}

\begin{table}[!h]
    \caption[Selected emission lines and temperatures]{
    Selected emission lines
    }
    \begin{center}
    \begin{tabular}{|l|l|l|} \hline
	Ion & Wavelength (\AA) & Peak formation temperature (MK) \\ \hline \hline
	Fe IX & 171 & 0.06 \\ \hline
	Fe X & 177 & 0.05  \\ \hline
	Fe XI & 180 & 0.04 \\ \hline
	Fe XII & 195 & 0.04 \\ \hline
	Fe XIII & 202 & 0.04 \\ \hline
	Fe XIV & 211 & 0.07 \\ \hline
	Fe XV & 284 & 0.08 \\ \hline
	Fe XVI & 335 & 0.17 \\ \hline
	Fe XVIII & 94 & 0.08 \\ \hline
	Fe XX & 132 & 0.20 \\ \hline
	\end{tabular}
    \\ \rule{0mm}{5mm}
    \end{center}
    \label{tab:emissionlines}
\end{table}

\begin{figure}[!h]
    \begin{center}
	    \includegraphics[width=166mm]{Images/AiaBandpasses.png}
    \end{center}
    \caption[AIA bandpasses]{
	    AIA bandpasses, model solar spectrum to provide an idea of the amount of blending. 
	}
    \label{aiabandpasses}
\end{figure}