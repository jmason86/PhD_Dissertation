\chapter{Introduction}
\label{chapterintro}

\begin{itemize}
    \item{} Solar eruptive events are rapid releases of energy on the Sun that are sometimes directed Earth-ward, making it important to understand them and to forecast their arrival time and magnitude of their impact
    \item{} Three basic types of eruptive event: flare, coronal mass ejection, energetic particles – this dissertation focuses on the first two
    \item{} Some background about solar flare prediction provided in Chapter 2, including my own massive statistical study, which went to print in ApJ my first year of graduate school
    \item{} The relationship between coronal mass ejections and the void they leave behind in the solar corona is the primary topic of the dissertation and its discussion spans several chapters. 
    \begin{itemize}
        \item{} Chapter 3 discusses the various physical processes that can lead to an observation that may be interpreted as a dimming, and the amalgamation of related observations that can theoretically be used to identify and isolate each mechanism
        \item{} Chapter 4 puts theory to the test in a detailed case study of a single, relatively simple, event. The aforementioned conglomeration of observations were used to determine that this was indeed a simple case with one dimming mechanism dominating the observation; that which theory says should be most strongly related to the associated CME
        \item{} Chapter 5 expands the study of the relationship between dimming and CMEs by performing an analysis similar to that of the case study but for approximately 30 events. Thus, a tentative statistical correlation between dimming and CME parameterizations could be derived. 
    \end{itemize}
    \item{} The topic of solar flares is picked up again briefly in the science motivation for the solar CubeSat MinXSS. An overview of the mission is the topic of Chapter 6, which includes science motivation, system overview, and lessons learned. 
    \item{} Chapter 7 delves deeper into the CubeSat engineering with a detailed thermal balance test and model analysis, culminating in the (likely) first ever tuned CubeSat thermal model that has been validated by dedicated testing and on-orbit measurements. 
    \item{} Chapter 8 provides a summary of deliverables and results, and lays out plans for future work. The latter will be the first steps in my post-doc that has already been secured through my first grant being funded as well as SDO/EVE and MinXSS extended mission funding. 
\end{itemize}


