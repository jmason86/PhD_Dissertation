\chapter{Introduction}
\label{chapterintro}

Solar eruptive events are among the most energetic phenomena in the solar system. As such, they power myriad physical processes that make the sun a highly dynamic environment -- an excellent natural laboratory for the study of high-energy and plasma physics, as well as for pushing the boundaries of remote sensing. The various processes are often cotemporal, which makes sorting out their influence on the electromagnetic spectrum nontrivial. Instruments with spatial resolution can alleviate some of this confusion, but those instruments often have relatively broad spectral resolution, which convolves the temperatures that are another critical piece of information for analyzing the events. The task before us requires creativity, thoroughness, and a good understanding of the advances already made. There is a practical motivation for studying solar eruptive events as well: sometimes they are directed toward the earth where they can have numerous impacts from the beautiful (e.g., the aurora) to the detrimental (e.g., satellite damage, radio communications interference, massive power disruption). 

The three basic types of solar eruptive event are solar flares, coronal mass ejections, and solar energetic particles. This dissertation focuses on the first two. Chapter \ref{chapterbackground} is dedicated to providing the context for the work in subsequent chapters. It first provides a tour of the sun from the core to the heliosphere, outlining the physics in each zone. Particular emphasis is placed on anything that can produce or influence photons because light is the observable that much of our understanding of the sun relies on, and the small contribution to that understanding made herein certainly relies on the interpretation of spectra. The chapter then delves deeper into the physics of solar eruptive events. In the broadest description, they consist of a long period of energy storage into the coronal magnetic field followed by the sudden and rapid release of that energy. The various pathways for energy release are particularly relevant for the later chapters and so solar flares and coronal mass ejections have dedicated subsections to provide more detail about them. Next, some of the space weather implications are described. The final section provides description of the instruments that are critical to the analyses of later chapters. 

Data from those instruments are turned to the purpose of characterizing the relationship between coronal mass ejections and the void they leave behind in the solar corona. That relationship and the physics surrounding it are the subject of Chapter \ref{chaptermechanisms}. Here, the various physical processes that can lead to the observation of a transient, localized dimming are described. Coronal mass ejections, i.e., mass loss, are only one possible way that coronal dimming could occur but the one most relevant for space weather. Competing thermal effects play an important role in the coronal irradiance. Obscuration of bright plasma by dark filaments, wave propagation, and Doppler effects can also have observational identifiers that potentially conflict with those from a departing coronal mass ejection. Fortunately, each identifier is somewhat unique, provided sufficient instrument spectral and/or spatial resolution. 

Chapter \ref{chaptercasestudy} applies the understanding of dimming gained from Chapter \ref{chaptermechanisms} to two event case studies. One was chosen for its relative simplicity: it only showed significant observational signatures from mass-loss dimming and thermal evolution. The other event was chosen for its complexity. It showed nearly all of the types of dimming described in Chapter \ref{chapterbackground}. In each case, the chapter first lays out the observations from a variety of instruments. A physically-motivated, emperical method is then developed for isolating and removing the influence of thermal evolution from irradiance such that mass-loss dimming can be more accurately measured with an irradiance (i.e., no spatial resolution) instrument. The light curves are then parameterized with the expectation that the slope of the light curve corresponds to the velocity of the coronal mass ejection and the depth of the dimming corresponds to the mass of the CME. These case studies do not provide the statistics necessary to establish whether or not those correlations exist. 

Chapter \ref{chapterstatistical} analyzes approximately 30 dimming events with associated coronal mass ejections in order to establish a relationship between their respective parameterizations. The process of event selection is detailed. Additionally, a study of the best functional fit to the light curves is presented followed by further discussion of the parameterization method. Finally, the positive correlations between dimming and coronal mass ejection parameterizations are described. 

The topic of solar flares is picked up again briefly in Chapter \ref{chapterminxssoverview}. This chapter provides an overview of the MinXSS CubeSat mission, including science motivation, system overview, and lessons learned. MinXSS is designed to measure the soft x-ray emission from the sun, much of which comes from the various physical processes that take place as part of a solar flare. My own contributions to this mission were varied, but at the time of writing MinXSS has only just begun taking observations. Thus, this chapter and the next have a stronger engineering tilt than the science focus of prior chapters. 

Chapter \ref{chapterthermal} delves into the details of thermal modeling, thermal balance testing, and model validation for MinXSS. Most CubeSat programs are not required to do thermal vacuum testing, which stresses the system to its operational and survival limits to ensure the spacecraft doesn't break under extreme conditions that it may experience on orbit. Thermal balance goes a step further and is correspondingly even less common in the CubeSat community. Its purpose is to validate the thermal model by putting the spacecraft in an environment that is as flight-like as possible. For example, one side of the vacuum chamber is hot while the rest of the chamber is cold. For a sun-pointing satellite like MinXSS, this is a good approximation of the sun on one side and deep space on all others. The chamber and measured conditions can be input into the thermal model, where the spacecraft thermal parameters can be tuned to match measured temperatures. In the case of MinXSS, the thermal performance is critical to the science because the sensor must be kept at -50 \degree C to prevent noise in the science data from drowning out the solar signal. 

Finally, Chapter \ref{chaptersummary} provides a summary of all results and makes suggestions for next steps. Of particular excitement to me, the work on coronal dimming suggests the possibility of new low-cost instruments to measure irradiance in a few key wavelengths, allowing the characterizion coronal mass ejections. Such an instrument could be leveraged in space weather to complement existing data or to provide a new and unique method of characterizing the coronal mass ejections of other stars. 
