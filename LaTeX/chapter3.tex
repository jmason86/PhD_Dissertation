\chapter{Mechanisms of Coronal Dimming}
\label{chaptermechanisms}

This chapter details the physics of coronal dimming. There are theoretically many physical processes that can lead to an uncareful observer identifying "dimming", which may have little to do with coronal mass ejection (CME). Traditionally, the term "coronal dimming" has been assumed to refer to the void left in the corona after a CME departs. This is one cause of a transient hole in the corona, and is of the greatest concern to space weather forecasters. Typically, a single dimming-sensitive wavelength or band will be observed and analyzed. However, changing temperatures, common during solar eruptive events, cause ionization fraction shifting, resulting in some emissions dimming while others brighten. Additionally, dark material (e.g., a filament) can pass between a bright region (e.g., flaring loops) and the observer, causing a transient dip in emission. Third, solar eruptive events sometimes have associated waves that propagate across the solar disk. These waves are observed as narrow bright fronts with a trailing dark region. The trailing dark region is another way to achieve a transient dimming of emission. Next, there are two ways that Doppler effects can cause transient dips in emission. The first is called Doppler dimming and results from fast moving plasma being sufficiently Doppler-shifted to reduce resonant fluorescence from the solar emission line sources; a phenomenon which is independent of the observation angle. The second occurs if eruptive plasma is moving fast enough in the line-of-sight to shift its emissions outside the bandpass of an observing instrument, which we have named "bandpass shift dimming". The physics, instrument effects, and mitigation strategies for each of these types of theoretically observable dimming are summarized in Table \ref{tablesummary} and are discussed in detail in the sections that follow.

\begin{table}[!htb]
    \caption{Summary of physical processes that can manifest as observed dimming}
    \begin{center}
    \begin{tabular}{|p{2.5cm}|p{4cm}|p{4cm}|p{4cm}|} \hline
	Short Name & Physical Process & EVE Observational Identifiers & AIA Observational Identifiers \\ \hline \hline
	Mass loss (Fig. \ref{figuremassloss}) & Ejection of emitting plasma from corona & Simultaneous intensity decrease in multiple coronal emission lines, with percentage decrease indicative of percentage mass lost & Area over and near the erupting active region (AR) darkens \\ \hline
	
	Thermal (Fig. \ref{figurethermal}) & Heating raises ionization states (e.g., a fraction of Fe IX becomes Fe X); cooling does the opposite & Heating: Emission loss in lines with lower peak formation temperatures and near simultaneous emission gain in lines with higher peak formation temperatures; vice versa for cooling & Heating: Area near AR darkens in channels with lower peak formation temperature and near simultaneous brightening in channels with higher peak formation temperatures; vice versa for cooling \\ \hline
	
	Obscuration (Fig. \ref{figureobscuration}) & Dim feature (e.g., filament material) moves into line-of-sight over a bright feature (e.g., flare arcade) & Drop of emission lines proportional to their absorption cross section in the obscuring material & Direct observation of this obscuration process \\ \hline
	
	Wave (Fig. \ref{figurewave}) & Wave disturbance propagates globally, causing compression/rarefaction of plasma as wave passes by & No effects have been identified & Direct observation of this wave process, especially apparent with difference movies \\ \hline
	
	Doppler & Fast moving plasma Doppler shifts away from resonant fluorescence with solar emission lines & Doppler wavelength shift of emission lines and change in intensity, possibly also observed as line broadening & Change in intensity of moving plasma as its velocity changes \\ \hline
	
	Bandpass & Emissions from fast moving plasma have Doppler wavelength shift & Emission line shifts in wavelength or has broadening & Doppler shift convolves with band-pass sensitivity to cause apparent reduction in emission \\ \hline
	
	\end{tabular}
    \\ \rule{0mm}{5mm}
    \end{center}
    \label{tablesummary}
\end{table}

\section{Thermal Dimming}
Temperature evolution of emission lines is only interpreted as observed dimming if one is not careful to observe co-spatial emission lines at different peak formation temperatures. As plasma is heated or cooled, the ionization fraction changes, necessarily causing the emission intensity to change (Figure \ref{thermalDimming}). For example, heating causes some Fe IX to become Fe X and thus, in the absence of competing physical processes, 171 \AA\ emission drops while 177 \AA\ rises. This pattern was identified observationally in Figure 6 of \citet{Woods2011} using SDO/EVE data, \citet{Robbrecht2010} using STEREO/EUVI, \citet{Jin2009} and \citet{Imada2007} with Hinode/EIS. It can also be observed in the standard composite (multi-wavelength) movies produced by the AIA team; indeed, this is one of the prime purposes for the composites. The initiation time and duration of temperature evolution tends to be quite similar to mass-loss dimming, as they are typically both responses to the rapid release of magnetic field energy in active regions and require several hours of recovery time. Thus, thermal processes could be mistaken for mass loss if only a single spectral line was observed. Ideally, unblended emission lines from an entire coronal ionization sequence (e.g., Fe I to Fe XVIII) could be used to mitigate this convolution of dimming observations. However, as we will show in Section 4.3, it may be sufficient to have observations of two sufficiently separated ionizations states to differentiate between thermal evolution and mass-loss dimming. This is due, in part, to the fact that hotter lines (e.g., Fe XV and above) are primarily emitted from confined loops near the flare and are thus not strongly impacted by mass-loss dimming. Multi-wavelength Doppler studies have shown that while all (measured) emission lines become blue-shifted (indicating an outflow), the magnitude of the shift is strongly directly proportional to the lines peak formation temperature \citep{Imada2007, Jin2009}. Figure \ref{upflowVsTemperature} shows this dependence for a plage region with a dimming during an X-class flare. In particular, Fe IX 171 \AA\ emission can be depressed further after open magnetic field lines from the departing CME close down and cause another bout of heating; causing e.g., Fe IX to become Fe X and beyond, which propagates outward as a "heat wave dimming" \citep{Robbrecht2010}. It may even be that cool emissions like Fe IX 171 \AA\ are simply moving too slow to account for mass depletion and that warmer lines, such as Fe XII 195 \AA\ better represent the mass being ejected \citep{Robbrecht2010}. However, \citet{Mason2014} found that the onset time, slope, and duration of dimming are comparable in SDO/AIA 171 \AA\ and 193 \AA\footnote{Note that the SDO/AIA 193 \AA\ band encompasses 195 \AA\ }and in SDO/EVE 171 \AA\ and 195 \AA\ (described in Chapter \ref{chaptercasestudy}). 

\begin{figure}[!htb]
	\caption{
	    Schematic depicting the observational difference between dimming and non-dimming emission
	    lines. Relative to a pre-eruption time (left), the Fe IX emission drops while the Fe XIV
	    emission increases (right) due to heating of the plasma and redistribution of ionization
	    states.
	}
    \begin{center}
	    \includegraphics[width=100mm]{Images/ThermalDimming.png}
    \end{center}
    \label{thermalDimming}
\end{figure}

\begin{figure}[!htb]
    \caption{
        Outflow velocity vs emission line peak formation temperature for a dimming region near a plage. 
        Adapted from \citet{Imada2007}.
        }
    \begin{center}
        \includegraphics[width=90mm]{Images/UpflowVsTemperature.png}
    \end{center}
    \label{upflowVsTemperature}    
\end{figure}


It is important to note that, in general, the magnitude of total observed dimming in a given line in EVE spectra is inversely proportional to its peak formation temperature, which can be inferred from Figure \ref{detectedDimmingVsIonization}. This figure was generated using a simple algorithm that searched all EVE/MEGS-A data for relative irradiance decreases greater than a specified threshold (1\%, 2\%, 3\%) of flares exceeding GOES X-ray class of C1. The window of time searched was bounded by the GOES event start time and the sooner of either 4 hours after the start time or the next GOES event start time. This algorithm was applied to all EVE data from mission start (2010 February 10) to the failure of the MEGS-A instrument (2014 May 26). MEGS-A takes the measurements of all wavelengths studied here. Figure \ref{detectedDimmingVsIonization} shows that the number of dimmings dramatically decreases as the magnitude threshold is increased, and decreases slightly with higher peak formation temperature. This latter effect is partially due to flare heating adding emission in the higher temperature, higher ionization state, lines that partially offsets the mass-loss dimming. Additionally, these trends indicate that at sufficiently high peak formation temperature, no dimming may be observed at all, even at the lowest detection threshold, which is consistent with the hotter lines being restricted to the confined flare loops and hence experiencing no mass loss. In other words, the higher the peak formation temperature, the greater the relative contribution of more confined loops to the measured emission. 

\begin{figure}[!h]
    \caption{
        Number of identified dimmings in EVE for six spectral lines using different percentage dimming depths as the
        threshold for a detection. There were 263 flares used to trigger an automated search for dimming in EVE. 
        Note the decrease in detections with increasing ionization state (i.e. peak formation temperature).
    }
    \begin{center}
        \includegraphics[width=150mm]{Images/DetectedDimmingVsIonization.png}
    \end{center}
    \label{detectedDimmingVsIonization}

\end{figure}

An instrument with spatial resolution like AIA can be used to isolate the confined flaring loops and create a time series of just the dimming region, and this is a procedure carried out in Chapter \ref{chaptercasestudy}. AIA too has its own limitations: relevant in this case is the relatively lower spectral resolution that blends together emission from several ionization states of Fe. With EVE and AIA combined, it is possible to analyze thermal dimming but the idea instrument for fully characterizing this phenomenon would be a high-resolution hyperspectral imager in the EUV.

\section{Obscuration Dimming}
The physical process that results in apparent dimming here is material that is dark in a particular wavelength (e.g., a filament) moving between bright material (e.g., flare arcade) and the observer (Figure \ref{obscurationDimming}). In optically thick wavelengths, the dark plasma absorbs some of the bright emission, resulting in an apparent decrease in emission. The slow draining of plasma back to the corona can obscure underlying emission for hours, and absorption can be observed in both coronal and chromospheric lines (e.g., \citet{Gilbert2013}). Although the obscuration dimmings can exhibit time and spatial scales comparable to the more short-lived mass-loss dimmings, it is fairly straightforward to identify absorption signatures in the EUV images. It may also be possible to identify this phenomenon with EVE using the He II 256 Å and 304 Å chromospheric emission lines and knowledge of the absorption cross-section through filamentary plasma. Figure \ref{photoionizationCrossSection} shows the photoionization cross-sections of the dominant species in the solar corona. Hydrogen and helium contribute an order-of-magnitude more absorption than metals\footnote{"Metals" in the astrophysical sense}, and thus the effect of metals can be ignored. The cross-sections are quite steep in the wavelength range of interest here (roughly 150-310 \AA). This means that approximately twice as much He II 256 Å than He II 304 Å emission will come through a filament. Furthermore, the mass-loss dimming sensitive lines (e.g., Fe IX 171 \AA\ and 195 \AA) will be less affected by this obscuration, but a 1\% effect would be sufficient to cause a "false" detection. It may be possible to identify obscuration dimming with EVE's 256 \AA\ and 304 \AA\ measurements and determine that an obscuration dimming has occurred. However, further analysis of this type of dimming is required before any conclusions can be drawn. 

\begin{figure}[!h]
    \caption{
        Schematic depicting the process of obscuration dimming. A filament previously obscuring only the quiet Sun (left)
        expands and moves in front of a flare arcade (right). This results in a decreased observed emission from the flare
        arcade in wavelengths where the filament is optically thick.
    }
    \begin{center}
        \includegraphics[width=75mm]{Images/ObscurationDimming.png}
    \end{center}
    \label{obscurationDimming}
\end{figure}

\begin{figure}[!h]
    \caption{
        Photoionization cross-sections for He I (dot-dashed line), He II (solid line), and H (dashed line) per hydrogen
        atom. The inset shows a wider wavelength range of the same data but with metals shown for comparison. The dashed
        vertical bars at the bottom indicate the edges of respective continua. The grey regions at the bottom are not 
        pertinent here as they correspond to specifics of the SOHO/CDS instrument. Adapted from \citet{Andretta2003}. 
    }
    \begin{center}
        \includegraphics[width=150mm]{Images/PhotoionizationCrossSection.png}
    \end{center}
    \label{photoionizationCrossSection}
\end{figure}


\section{Wave Dimming}
\label{sec:waveDimming}

\section{Two Possible but Unobserved Dimming Mechanisms}

\section{Mass-loss Dimming}

